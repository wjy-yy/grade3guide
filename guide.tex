\documentclass[UTF8,11pt,a4paper]{ctexart}
\usepackage{hyperref}
\usepackage{amsmath}
\usepackage{geometry}
\usepackage{ulem}
\usepackage{multirow}
\usepackage{fontspec}
\usepackage{graphicx}
\usepackage{amsfonts}
\usepackage{amssymb}
\usepackage{booktabs}
\usepackage{fancyhdr}

\geometry{left=2cm,right=2cm,top=2cm,bottom=2.0cm}
\title{\textbf{高三生存指南}}

\author{武汉大学 wjyyy}

\pagestyle{fancy}

\fancyhead[L]{高三生存指南}
\fancyhead[R]{wjyyy}
\fancyfoot[C]{第 \thepage 页,共 7 页}

\setmonofont{Consolas}

\begin{document}
	\maketitle
	\tableofcontents
	\section{高三概况}
		高三一年,你会如同一个学习机器一样,每天从早学到晚,任何放松都可能会使你感到罪恶。用郦道元的一句环境描写借景抒情就是:
		\begin{quote}
			\textit{自非亭午夜分,不见曦月。}
			\flushright\textit{——郦道元《三峡》}
		\end{quote}
		
		高三会有各种名目的考试,从强化训练、日常考试、周考、月考、联考、调考、……直到高考。总会有订正不完的错题、分析不完的试卷、以及出奇不意的新题型、新概念。
		
		刚上高三的时候,会面对一轮复习,持续约4个月,接着有二轮三轮复习。如果高一高二不是很努力,不用担心,多轮的高效复习会帮助你查漏补缺,但是必须付出一些高一高二没有付出的时间和精力。
		
		高三的寒假是绝佳的弯道超车机会,因为它为你提供了足够的时间来做题、落实、总结。
		
		高考前,学校会组织多次适应性考试,来让考生适应高考的氛围。就我而言,与其说是适应高考,还不如说是在让学生对高考的紧张感到麻木,不再紧张地踏上高考考场。
		
		\begin{quote}
			\textit{哎呀,又周二了,背会书去考试。}
			\flushright\textit{——wjyyy于2020年7月7日}
		\end{quote}
	
		因为2020年的高考是周二周三,所以6月的每个周二周三,学校都组织了高质量的周考。5周以后,我们已经养成了周二参加周考的“习惯”。
		
		高考结束后14天,可以通过准考证上提供的途径查询分数,随后可以填报志愿。
	\section{高三建议 - 方法}
	
	\section{高三建议 - 学科}
	
	\section{高三建议 - 态度}%乐观
\end{document}